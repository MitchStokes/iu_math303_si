\documentclass[11pt]{exam}
\usepackage{../commonheader}
\usepackage{graphicx}

\title{Vector Span, Linearly Independent Vectors}
\date{Week 5, Session 1}

\begin{document}
\maketitle

\section{Opening}

\vspace{20px}
\begin{questions}
    \item What is the determinant of the following matrix, $A$?
    $$A = \begin{bmatrix} 2 & 3 & 3 \\ 0 & 4 & 2 \\ 0 & 0 & -2 \end{bmatrix}$$
    \item For the same matrix, what is $det(\dfrac{1}{2}A)$?
    \item Is $A$ invertible?
\end{questions}

\vspace{20px}
\section{Span of a Vector Set}
    The span of a set of vectors $\{ \vec{u_1}, \dots, \vec{u_k} \}$ in $\mathbb{R}^n$ is known as the span of these vectors, and
    is written as $span\{ \vec{u_1}, \dots, \vec{u_k} \}$.

    In words, if we have a set of vectors $\{ \vec{u_1}, \dots, \vec{u_k} \}$, any vector $\vec{v}$ for which there exists scalars
    $k_1, k_2, \dots, k_k$ such that $\vec{v} = k_1 \vec{u_1} + k_2 \vec{u_2} + \dots + k_k \vec{u_k}$,
    then $v \in span\{ \vec{u_1}, \dots, \vec{u_k} \}$.

    \begin{questions}
        \item Describe the span of the following sets of vectors in $\mathbb{R}^2$:
        \begin{enumerate}
            \item $\{ [1,0]^T \}$
            \item $\{ [1,1]^T \}$
            \item $\{ [1,0]^T, [0,1]^T \}$
            \item $\{ [-1,0]^T, [1,0]^T \}$
            \item $\{ [2,2]^T, [-1,-1]^T \}$
        \end{enumerate}

        \item Suppose we have a set of vectors $\{ \vec{u}, \vec{v}, \vec{w} \}$ such that $\vec{u}$ and $\vec{v}$ span the entire XY-plane,
        and that $\vec{w} = [2,-5,0]$. What is the span of this set in $\mathbb{R}^3$, and why?
        \item Suppose we have a set of vectors $\{ \vec{u}, \vec{v}, \vec{w} \}$ such that $\vec{u}$ and $\vec{v}$ span the entire XY-plane,
        and that $\vec{w} = [2, 2, 1]$. What is the span of this set in $\mathbb{R}^3$, and why?
    \end{questions}

    \vspace{20px}
    So how do we show whether a given vector is in the span of a set of vectors? Let's consider a set of two vectors consisting of
    $\vec{u} = [1,1,0]^T$ and $\vec{v} = [3,2,0]^T$, and a third vector $\vec{w} = [4,5,0]^T$.

    To have $\vec{w} \in span\{ \vec{u}, \vec{v} \}$, there must exist some scalars $a,b$ such that:
    $$\vec{w} = a \vec{u} + b \vec{v}$$
    You might realize that this generates a new system for us to solve:
    $$\begin{bmatrix} 4 \\ 5 \\ 0 \end{bmatrix} = a \begin{bmatrix} 1 \\ 1 \\ 0 \end{bmatrix} + b \begin{bmatrix} 3 \\ 2 \\ 0 \end{bmatrix}$$
    $$a + 3b = 4; a + 2b = 5$$
    We can solve this system like usual:
    $$\begin{amatrix}{2} 1 & 3 & 4 \\ 1 & 2 & 5 \end{amatrix} \Rightarrow \begin{amatrix}{2} 1 & 0 & 7 \\ 0 & 1 & -1 \end{amatrix}$$
    So, the solution is $a = 7, b = -1$. This means that $\vec{w} = 7\vec{u} - \vec{v}$.

    Since we found a solution $a,b$ for this new system, we know that there exists some linear combination of $\vec{u}, \vec{v}$ that equals $\vec{w}$.

    \vspace{20px}
    \begin{questions}
        \item In the example above, how would we be able to tell if $\vec{w} \notin span\{ \vec{u}, \vec{v} \}$? Try to come up with a choice of $\vec{w}$
        that isn't in the span, and show that it isn't in the span using the augmented matrix.
    \end{questions}

\pagebreak
\section{Linear Independence}
    
    \vspace{20px}
    \subsection{Conditions for Linear Dependence/Independence}
    A set of non-zero vectors is linearly dependent if there exists scalars $k_1, k_2, \dots k_n$ (where at least one $k_i$ is nonzero),
    such that the linear combination of said vectors using those scalars equals the zero vector.

    You may notice that this is similar to homogeneous systems!

    There are three equivalent conditions for linear independence:
    \begin{enumerate}
        \item The inverse of the statement above. That is, the system is linearly \textit{independent} if there doesn't exist such a choice
        of scalars $k_1, k_2, \dots k_n$.
        \item No vector in the set is in the span of the others.
        \item The system of linear equations $A \vec{x} = \vec{0}$ has only the trivial solution, where $A$ is the matrix made up by using
        the set of vectors as columns.

        \begin{questions}
            \item Is the following set of vectors linearly independent?
            $$\{ [1,1,0]^T, [2,3,1]^T, [-1,1,2]^T \}$$
            \item Let's consider the second condition from above. Suppose we have a set of vectors $\{ \vec{u}, \vec{v}, \vec{w} \}$ where
            $\vec{w} \in span \{ \vec{u}, \vec{v} \}$. Prove that there exists a set of scalars $k_1, k_2, k_3$ such that
            $\vec{0} = k_1 \vec{u} + k_2 \vec{v} + k_3 \vec{w}$.
            \item Suppose we have a set of 4 vectors in $\mathbb{R}^3$. Can this set be linearly independent? Why or why not?
        \end{questions}

        \vspace{20px}
        Let $U \subseteq \mathbb{R}^n$ be a linearly independent set of vectors. Then any vector in $span(U)$ can be written
        \textit{uniquely} as a linear combination of vectors of $U$.

        \vspace{20px}
        When an $n$-element set of vectors in $\mathbb{R}^n$ makes up an invertible $n \times n$ matrix $A$, the vectors are independent and
        span $\mathbb{R}^n$. The rows of $A$ are thus also independent.

        \begin{questions}
            \item Why is it true that the columns of an invertible matrix are linearly independent?
        \end{questions}
    \end{enumerate}

    \pagebreak
    \section{Closing}
    \begin{questions}
        \item What is the largest $n$ for which a set of 4 vectors can span $\mathbb{R}^n$?
        \item Are the columns of the following matrix linearly independent? What is the span of the columns?
        $$\begin{bmatrix} 1 & 0 & 0 \\ 2 & 3 & 4 \\ 2 & 2 & 2 \end{bmatrix}$$
    \end{questions}

\end{document}
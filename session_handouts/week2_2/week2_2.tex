\documentclass[11pt]{exam}
\usepackage{../commonheader}
\usepackage{graphicx}

\title{Matrix Multiplication \& Transpose}
\date{Week 2, Session 2}

\begin{document}
\maketitle

\section{Opener: Pop Quiz on Content so Far}
    \begin{questions}
        \item A 3-variable linear homogeneous system has rank 2. How many free variables does it have? How many solutions does it have?
        \item \textbf{Challenge question: } A 3-variable linear homogeneous system has rank 2. How many solutions does it have?
        \item How many basic solutions does a system with two free variables have?
        \item Sum the following two matrices:
        $$\begin{bmatrix} 0 & -1 \\ 1 & -2 \end{bmatrix} + \begin{bmatrix} 1 & 2 \\ 3 & 4 \end{bmatrix}$$
        \item Sum the following two matrices:
        $$\begin{bmatrix} 0 & -1 \\ 1 & -2 \end{bmatrix} + \begin{bmatrix} 2 & 2 \\ 3 & 2 \\ 4 & 5\end{bmatrix}$$
    \end{questions}

\pagebreak
\section{Vectors}
    
    \vspace{20px}
    \subsection{Row and Column Vectors}
    \begin{questions}
        \item Determine whether the following matrices are \textit{row vectors}, \textit{column vectors}, or neither.
        \begin{enumerate}[a.]
            \item $\begin{bmatrix} 1 \\ 0 \end{bmatrix}$
            \item $\begin{bmatrix} 1 & 2 \\ 2 & 0 \end{bmatrix}$
            \item $\begin{bmatrix} 1 & 2 \end{bmatrix}$
        \end{enumerate}
    \end{questions}

    \vspace{20px}
    \subsection{Vector Form of a System of Linear Equations}
        Recall that linear systems of $m$ equations in $n$ variables have the form:
        $$a_{11}x_1 + a_{12}x_2 + \dots + a_{1n}x_n = b_1$$
        $$a_{21}x_1 + a_{22}x_2 + \dots + a_{2n}x_n = b_2$$
        $$\vdots$$
        $$a_{m1}x_1 + a_{m2}x_2 + \dots + a_{mn}x_n = b_3$$

        It turns out that we can write this system in a more convenient \textbf{vector form}:
        $$x_1 \begin{bmatrix} a_{11} \\ a_{21} \\ \vdots \\ a_{m1} \end{bmatrix} +
          x_2 \begin{bmatrix} a_{12} \\ a_{22} \\ \vdots \\ a_{m2} \end{bmatrix} +
          \dots +
          x_n \begin{bmatrix} a_{1n} \\ a_{2n} \\ \vdots \\ a_{mn} \end{bmatrix} =
          \begin{bmatrix} b_1 \\ b_2 \\ \vdots \\ b_m \end{bmatrix}
          $$

        See that each vector would be one column from the augmented matrix, and the column on the right-hand side of the equation
        would be the "augmented" column on the far right.

    \vspace{20px}
    \subsection{Multiplication of Vector by Matrix}
        Consider an $m \times n$ matrix $A = \begin{bmatrix} A_1 \dots A_n \end{bmatrix}$
        and an $n$-dimensional column vector $\vec{x} = \begin{bmatrix} x_1 \\ \vdots \\ x_n \end{bmatrix}$.

        The product $A \vec{x}$ is an $m \times 1$ column vector which can be written as a linear combination of the columns of $A$, similar
        to the vector form above!
        $$A \vec{x} = x_1A_1 + x_2A_2 + \dots + x_nA_n$$

        \pagebreak
        \begin{questions}
            \item Perform the following matrix-vector multiplications by decomposing the problem into vector form:
            \begin{enumerate}
                \item $\begin{bmatrix} 1 & 1 \\ 1 & 1 \end{bmatrix} \begin{bmatrix} 1 \\ 1 \end{bmatrix} =$
                \item $\begin{bmatrix} 3 & 2 \\ 1 & 2 \end{bmatrix} \begin{bmatrix} 0 \\ 1 \end{bmatrix} =$
                \item $\begin{bmatrix} 0 & 2 \\ 1 & 0 \end{bmatrix} \begin{bmatrix} 2 \\ 1 \end{bmatrix} =$
                \item $\begin{bmatrix} 0 & 2 \\ 1 & 0 \\ 3 & 3 \end{bmatrix} \begin{bmatrix} 2 \\ 1 \end{bmatrix} =$
            \end{enumerate}
        \end{questions}

    \vspace{20px}
    \subsection{Matrix Form of a System of Linear Equations}
    We can also write a linear system in \textbf{matrix form}, taking advantage of the matrix-vector multiplication we just learned!
    $$\begin{bmatrix}
        a_{11} & a_{12} & \dots & a_{1n} \\
        a_{21} & a_{22} & \dots & a_{2n} \\
        \vdots & \vdots & \ddots & \vdots \\
        a_{m1} & a_{m2} & \dots & a_{mn}
    \end{bmatrix}
    \begin{bmatrix} x_1 \\ x_2 \\ \vdots \\ x_n \end{bmatrix} =
    \begin{bmatrix} b_1 \\ b_2 \\ \vdots \\ b_m \end{bmatrix}$$
    When you decompose the vector multiplication like we did before, you get the vector form:
    $$x_1 \begin{bmatrix} a_{11} \\ a_{21} \\ \vdots \\ a_{m1} \end{bmatrix} +
      x_2 \begin{bmatrix} a_{12} \\ a_{22} \\ \vdots \\ a_{m2} \end{bmatrix} +
      \dots +
      x_n \begin{bmatrix} a_{1n} \\ a_{2n} \\ \vdots \\ a_{mn} \end{bmatrix} =
      \begin{bmatrix} b_1 \\ b_2 \\ \vdots \\ b_m \end{bmatrix}$$
    So these two forms are indeed interchangeable!

    The vector $\begin{bmatrix} x_1 \\ x_2 \\ \vdots \\ x_n \end{bmatrix}$ will be a solution to the system if and only if
    $x_1, x_2, \dots, x_n$ are solutions to the linear system represented in either form.

\pagebreak
\section{Matrix Multiplication}
    Similarly, we can multipliy two matrices together. Consider the following example:
    $$A = \begin{bmatrix} 1 & 2 & 1 \\ 0 & 2 & 1 \end{bmatrix}, B = \begin{bmatrix} 1 & 2 & 0 \\ 0 & 3 & 1 \\ -2 & 1 & 1 \end{bmatrix}$$

    First, we need to check the size of the matrices to make sure multiplication is possible. In this case, $A$ is $2 \times 3$, and
    $B$ is $3 \times 3$. The trick to remember is that when you write the sizes side-by-side \textit{in the same order as the multiplication}, 
    like $(2 \times \boxed{3})(\boxed{3} \times 3)$, you need the "inner" (boxed) entries to be equal. If they're not, the multiplication isn't possible.

    \begin{questions}
        \item Determine whether the following matrix sizes are compatible for multiplication:
        \begin{enumerate}
            \item $1 \times 2$ and $2 \times 1$
            \item $3 \times 3$ and $3 \times 4$
            \item $2 \times 2$ and $4 \times 2$
            \item $1 \times 1$ and $1 \times 2$
        \end{enumerate}
        \item Does the fact that a multiplication works in one direction necessarily imply that it works in the other direction?
    \end{questions}

    Once you've determined two matrices are compatible, we can decompose the problem further into matrix-vector form like so:
    $$\begin{bmatrix}
        \begin{bmatrix} 1 & 2 & 1 \\ 0 & 2 & 1 \end{bmatrix}\begin{bmatrix} 1 \\ 0 \\ -2 \end{bmatrix},
        \begin{bmatrix} 1 & 2 & 1 \\ 0 & 2 & 1 \end{bmatrix}\begin{bmatrix} 2 \\ 3 \\ 1 \end{bmatrix},
        \begin{bmatrix} 1 & 2 & 1 \\ 0 & 2 & 1 \end{bmatrix}\begin{bmatrix} 0 \\ 1 \\ 1 \end{bmatrix},
    \end{bmatrix}$$
    That is, the $n$th column of the resulting matrix will be the column vector created by multiplying $A$ with the $n$th column of $B$.

    \begin{questions}
        \item Perform the following matrix multiplication using the column decomposition method described above:
        $$\begin{bmatrix} 2 & 3 \\ 3 & 4 \end{bmatrix} \begin{bmatrix} 1 & 1 \\ 2 & 1 \end{bmatrix}$$
        \item Perform the matrix multiplication from the example above.
    \end{questions}

    We can also consider matrix multiplication on an entry-by-entry basis (live demo to explain).

    \begin{questions}
        \item Perform the following matrix multiplication using element-by-element multiplication:
        $$\begin{bmatrix} 1 & 3 \\ 4 & 3 \end{bmatrix} \begin{bmatrix} 1 & 2 \\ 2 & 2 \end{bmatrix}$$
    \end{questions}

\pagebreak
\section{Transpose}

    \vspace{20px}
    \subsection{Vector/Matrix Transposition}
        The transpose of a vector is the same vector, but "flipped" in direction. That is, a row vector becomes a column vector with the same
        entries, and vice versa. For example:
        $$\begin{bmatrix} 1 \\ 2 \\ 3 \end{bmatrix}^T = \begin{bmatrix} 1 & 2 & 3 \end{bmatrix}$$

        \begin{questions}
            \item Find the following vector tranpositions:
            \begin{enumerate}
                \item $\begin{bmatrix} 0 \\ 2 \\ 2 \end{bmatrix}^T$
                \item $\begin{bmatrix} 1 & 3 & 3 & 2 \end{bmatrix}^T$
                \item $\begin{bmatrix} 2 \\ 2 \end{bmatrix}^T$
            \end{enumerate}
        \end{questions}

        For a matrix, the $(i,j)$th entry of $A$ becomes the $(j,i)$th entry of $A^T$. In practice, this means the 1st row becomes the 1st column,
        the 2nd row becomes the 2nd column, etc. See the following example (first row highlighed):
        $$\begin{bmatrix} \boxed{1} & \boxed{2} & \boxed{3} \\ 4 & 5 & 6 \end{bmatrix}^T =
        \begin{bmatrix} \boxed{1} & 4 \\ \boxed{2} & 5 \\ \boxed{3} & 6 \end{bmatrix}$$

        \begin{questions}
            \item I think the easiest way to understand this is to drill some practice problems, so try the following:
            \begin{enumerate}
                \item $\begin{bmatrix} 1 & 2 \\ 3 & 4 \end{bmatrix}^T$ =
                \item $\begin{bmatrix} 1 & 2 \\ 3 & 4 \\ 5 & 6 \end{bmatrix}^T$ =
                \item $\begin{bmatrix} 1 & 2 & -1 \\ 2 & 4 & -3 \\ 1 & 5 & 6 \end{bmatrix}^T$ =
            \end{enumerate}
        \end{questions}

    \pagebreak
    \subsection{Properties of the Transpose}
        Let $A$ be an $m \times n$ matrix, $B$ an $n \times p$ matrix, and $r, s$ scalars. Then:
        \begin{enumerate}[a.]
            \item $(A^T)^T = A$
            \item $(AB)^T = B^T A^T$
            \item $(rA + sB)^T = rA^T + sB^T$
        \end{enumerate}
        Additionally, a matrix is said to be \textbf{symmetric} if $A = A^T$. It is said to be \textbf{skew symmetric} if $A = -A^T$.

        \begin{questions}
            \item Is the following matrix symmetric?
            $$\begin{bmatrix}
                1 & 2 & 3 \\ 2 & 2 & 2 \\ 3 & 2 & 1
            \end{bmatrix}$$
        \end{questions}

    \pagebreak
    \subsection{Closing}
    Let's finish off the session with some summarizing questions!
    \begin{questions}
        \item Perform the following matrix multiplication:
        $$\begin{bmatrix} 2 & 3 \\ 4 & -1 \end{bmatrix} \begin{bmatrix} 1 & -1 \\ -1 & -1 \end{bmatrix}$$
        \item When writing a linear system in $A\vec{x} = \vec{b}$ form, are the $\vec{x}$ and $\vec{b}$ vectors \textit{row} or \textit{column} vectors?
        \item Write the following linear system in equation form:
        $$\begin{bmatrix} 1 & 2 \\ -1 & 4 \end{bmatrix} \begin{bmatrix} x \\ y \end{bmatrix} = \begin{bmatrix} 1 \\ 2 \end{bmatrix}$$
        \item Without doing any calculations, what is the solution to the following system?
        $$\begin{bmatrix} 2 & 0 \\ 0 & 1 \end{bmatrix} \begin{bmatrix} x \\ y \end{bmatrix} = \begin{bmatrix} 2 \\ 2 \end{bmatrix}$$
    \end{questions}
    

\end{document}
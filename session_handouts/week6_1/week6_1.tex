\documentclass[11pt]{exam}
\usepackage{../commonheader}
\usepackage{graphicx}

\title{Orthogonality}
\date{Week 6, Session 1}

\begin{document}
\maketitle

\section{Opener}
    \vspace{20px}
    \begin{questions}
        \item Suppose a $4 \times 4$ matrix $U$ has $dim(row(U)) = 3$. What is its rank?
        \item Suppose a $4 \times 4$ matrix $U$ has $dim(row(U)) = 1$. What is its nullity (the dimension of its null space)?
        \item Loosely describe the process to find the null space of a matrix.
    \end{questions}

\vspace{40px}
\section{Orthogonal Vectors}
    
    \vspace{20px}
    \subsection{Conditions for Orthogonality}
    Two vectors $\vec{u}, \vec{v}$ are orthogonal if their dot product is 0:
    $$\vec{u} \cdot \vec{v} = 0$$
    Subsequently, an \textit{orthogonal set of vectors} is a set $\{ \vec{u_1}, \vec{u_2}, \dots, \vec{u_k} \}$ where each vector is mutually orthogonal:
    \begin{enumerate}
        \item $\vec{u_j} \cdot \vec{u_j} = 0$ if $i \neq j$
        \item $\vec{u_1} \neq \vec{0}$ for all $i$
    \end{enumerate}
    Additionally, the set is orthonormal if each such dot product is either 0 or 1 (when $i = j$).

    Also, orthogonal vectors are definitionally linearly independent!

    We can normalize an orthonormal set by dividing each vector by its magnitude. For example:
    $$ \left \{ \dfrac{1}{||\vec{u_1}||} \vec{u_1},  \dfrac{1}{||\vec{u_2}||} \vec{u_2}, \dots, \dfrac{1}{||\vec{u_k}||} \vec{u_k} \right \}$$
    This is just changing the vector to point in the same direction but with its magnitude coerced to 1.

    \vspace{20px}
    \begin{questions}
        \item If the dot product of a vector times itself is 1, what does this mean in practice?
        \item Suppose $\{ \vec{u_1}, \vec{u_2} \} = \{ [1,1]^T, [-1, 1]^T \}$. Find whether this set is orthogonal and/or orthonormal.
    \end{questions}
    

\pagebreak
\section{Orthogonal Matrices}
    
    \vspace{20px}
    An $n \times n$ matrix $U$ is orthogonal if $U U^T = U^T U = I$. Since $U$ is assumed to be square, it's enough to check that one of these
    equalities to $I$ holds.

    It turns out that the rows of an $n \times n$ orthogonal matrix form an \textit{orthonormal} basis of $\mathbb{R}^n$! Similarly, you can construct
    an $n \times n$ orthogonal matrix using an orthonormal basis of $\mathbb{R}^n$.

    The determinant of an orthogonal matrix is $\pm 1$.

    \vspace{20px}
    \begin{questions}
        \item Is the matrix $U = \begin{bmatrix} 1 & 0 & 0 \\ 0 & 0 & -1 \\ 0 & -1 & 0 \end{bmatrix}$ orthogonal? What is its determinant?
    \end{questions}

\pagebreak
\section{Gram-Schmidt Process}
    \vspace{20px}
    (Live walkthrough using textbook)

    \vspace{20px}
    \begin{questions}
        \item Find an orthogonal basis for the space spanned by $[-1, -2, 1]^T$ and $[0,1,-2]^T$.
        \item Find an orthogonal basis for the column space of $\begin{bmatrix} 2 & 1 \\ 2 & 1 \\ 6 & 0 \end{bmatrix}$.
        \item Find an orthonormal basis for the column space of $\begin{bmatrix} 2 & -1 \\ 8 & -1 \\ 0 & 2 \end{bmatrix}$.
    \end{questions}

\pagebreak
\section{Closing}
Let's finish up with a few conceptual questions!
\begin{questions}
    \item Suppose we have a set of three vectors with magnitudes of 1 in $\mathbb{R}^n$ such that $\vec{u_1}$ lies along the x-axis,
    $\vec{u_2}$ lies along the z-axis, and $\vec{u_3}$ lies along the y-axis. What space do these vectors form a basis of?
    Is the basis orthogonal? Orthonormal?
    \item Suppose we construct a matrix using the same vectors, $U = \begin{bmatrix} \vec{u_1}^T \\ \vec{u_2}^T \\ \vec{u_3}^T \end{bmatrix}$. What
    is the determinant of this matrix?
\end{questions}

\end{document}
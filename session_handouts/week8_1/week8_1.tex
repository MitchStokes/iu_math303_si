\documentclass[11pt]{exam}
\usepackage{../commonheader}
\usepackage{graphicx}

\title{Diagonalization}
\date{Week 8, Session 1}

\begin{document}
\maketitle

\section{Opening: Eigenvalues and Eigenvectors}
\vspace{20px}
    \begin{questions}
        \item Find the eigenvalues and eigenvectors of $A = \begin{bmatrix} -1 & -4 \\ -3 & -2 \end{bmatrix}$.
        \item Suppose a matrix $A$ is known to have an eigenvalue $\lambda = 3$
        with associated eigenvector $\vec{x} = \begin{bmatrix} 3 \\ 2 \end{bmatrix}$.
        Solve for $\vec{y} = A\begin{bmatrix} -6 \\ -4 \end{bmatrix}$.
    \end{questions}

\pagebreak
\section{Similarity}
    \vspace{20px}
    Two matrices $A,B$ are similar if and only if we can find some invertible matrix $P$ such that:
    $$A = P^{-1}BP$$
    Similar matrices have the same determinants, rank, trace, and eigenvalues.

    When a matrix is similar to a diagonal matrix, we say that the matrix is \textit{diagonalizable}.

\vspace{50px}
\section{Diagonalization}
    \vspace{20px}
    When diagonalizing a matrix $A$, we're trying to find three other matrices $P, D,$ and $P^{-1}$ such that $A = PDP^{-1}$, and $D$ is a matrix
    with nonzero entries \textit{only} on its diagonal.

    A matrix $A$ is diagonalizable if and only if there is an invertible matrix $P$ given by:
    $$P = \begin{bmatrix} X_1 X_2 \dots X_n \end{bmatrix}$$
    where the $X_k$ are the eigenvectors of $A$. Additionally, the matrix $D$ constructed by $A = PDP^{-1}$ will have diagonal
    entries consisting of the corresponding eigenvalues of $A$.

    \vspace{20px}
    \begin{questions}
        \item Let $A = \begin{bmatrix} 2 & 0 & 0 \\ 1 & 4 & -1 \\ -2 & -4 & 4 \end{bmatrix}$. Find an invertible matrix $P$ and a
        diagonal matrix $D$ such that $P^{-1}AP = D$.
    \end{questions}

    \vspace{20px}
    \subsection{Conditions for Diagonalization}
    It turns out that we can always diagonalize an $n \times n$ matrix if it has $n$ distinct eigenvectors.
    In practice, this means that if we have $n$ different eigenvalues, we can diagonalize. If the number of eigenvalues
    if less than $n$ (implying that some eigenvalue(s) have multiplicity $>1$), we may still be able to diagonalize if each
    eigenvalue of multiplicity $m$ corresponds to $m$ different eigenvectors.
    
    In other words, if $A$ has eigenvalues with multiplicity $> 1$, that eigenvalue needs to have as many eigenvectors associated with it
    as its multiplicity. If this is the case for all eigenvalues, the matrix will still be diagonalizable.

    \pagebreak
    \subsection{A Useful Property of Diagonal Matrices}
    One useful case for diagonal matrices is when multiplying a matrix by itself some number of times (i.e. raising it to a power).

    Consider trying to find $A^{100}$. This would be very annoying to calculate by hand, since you'd be perfoming the matrix multiplication 100 times!

    Instead, you could diagonalize $A$ into $A = PDP^{-1}$ and raise this representation to the 100th power.
    This results in $A^{100} = (PDP^{-1})^{100} = PD^{100}P^{-1}$.
    Since $D$ is a matrix with only entries on the diagonal, $D^{100}$ is very easy to find; just raise each nonzero entry to the 100th power!

\vspace{50px}
\section{Closing/Practice Problems}
\begin{questions}
    \item Diagonalize (find $P, D, P^{-1}$) for the following matrix:
    $$A = \begin{bmatrix} 3 & 0 & 0 \\ -3 & 4 & 9 \\ 0 & 0 & 3 \end{bmatrix}$$
    \item Use diagonalization to find $A^4$ for $A = \begin{bmatrix} 1 & -6 \\ 2 & -6 \end{bmatrix}$.
\end{questions}

\end{document}
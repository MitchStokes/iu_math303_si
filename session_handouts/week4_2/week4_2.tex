\documentclass[11pt]{exam}
\usepackage{../commonheader}
\usepackage{graphicx}

\title{Properties of Determinants \& Cramer's Rule}
\date{Week 4, Session 2}

\begin{document}
\maketitle

\section{Opener: Midterm Exam Recap}

By now, I think you all would've taken the midterm. With that in mind, let's take a second to reflect on the course so far. Take a
few minutes, and think quietly about your responses to the following prompts. We'll take a few minutes to discuss as a group afterwards!
\begin{enumerate}
    \item What is one concept from the midterm (or course so far) that you wish you understood better?
    \item Similarly, what's one concept that you feel particularly strong about; especially if it's something difficult that you feel
    proud of yourself for having learned?
    \item On a scale from 1 to 10, how effective do you think your study strategy has been for the course so far? What's one strategy
    you've used that you've found to be successful?
\end{enumerate}


\pagebreak
\section{Properies of Determinants}
    
    \vspace{20px}
    \subsection{Determinants and Row Operations}
    Recall that there are three row operations for matrices:
    \begin{enumerate}
        \item Switching or interchanging two rows
        \item Multiplying a single row by a scalar
        \item Adding a scalar multiple of one row to another row
    \end{enumerate}
    These row operations each have different mutating effects on the determinant of a matrix. Consider a matrix $A$ with determinant
    $det(A)$. Consider a matrix $B$ which is created by performing each of the above row operations on $A$. Then:
    \begin{enumerate}
        \item (Switching rows) $det(B) = -det(A)$
        \item (Scalar multiple of a row) $det(B) = k \cdot det(A)$, where $k$ is the scalar multiple used
        \item (Adding a scalar multiple of a row to another) $det(B) = det(A)$
    \end{enumerate}

    \vspace{20px}
    For the following problems, suppose $A = \begin{bmatrix} 1 & 0 & 0 \\ 2 & 3 & 4 \\ 2 & 4 & 3 \end{bmatrix}$.
    \begin{questions}
        \item What is $det(A)$?
        \item Suppose $B = \begin{bmatrix} 1 & 0 & 0 \\ 2 & 4 & 3 \\ 2 & 3 & 4 \end{bmatrix}$. Without manual calculation, what is
        $det(B)$?
        \item Suppose $B = \begin{bmatrix} 1 & 0 & 0 \\ 5 & 3 & 4 \\ 2 & 4 & 3 \end{bmatrix}$. Without manual calculation, what is
        $det(B)$?
        \item Suppose $B = \begin{bmatrix} 1 & 0 & 0 \\ 6 & 9 & 12 \\ 2 & 4 & 3 \end{bmatrix}$. Without manual calculation, what is
        $det(B)$?
        \pagebreak
        \item Suppose $B = \begin{bmatrix} 1 & 0 & 0 \\ 5 & 6 & 8 \\ 2 & 4 & 3 \end{bmatrix}$. Without manual calculation, what is
        $det(B)$? \textit{Hint: Multiple row operations were applied to row 2.}
        \item Suppose $B = 2A$. Without manual calculation, what is $det(B)$?
    \end{questions}

    \vspace{20px}
    \subsection{Other Properties of Determinants}

    As we saw above, multiplying a matrix by a scalar $k$ is the same as multiplying each row of the matrix by $k$. Since each
    row multiplication multiplies the determinant by $k$, for an $n \times n$ matrix you will be performing the multiplication
    $n$ times. As such, the determinant will cumulatively be multiplied by $k^n$.

    In other words for an $n \times n$ matrix $A$, $det(kA) = k^n det(A)$.

    There are several other properties of determinants we'll need to know; suppose we have two matrices $A$ and $B$. Then:
    \begin{enumerate}
        \item $det(AB) = det(A)det(B)$
        \item $det(A^T) = det(A)$
        \item $det(A^{-1}) = \dfrac{1}{det(A)}$ (when $det(A) \neq 0$)
    \end{enumerate}

\pagebreak
\section{Cramer's Rule}
    
    \vspace{20px}
    Suppose we have a linear system of the form $A\vec{x} = \vec{b}$, where $A$ is a square invertible matrix.
    Then $\vec{x} = \begin{bmatrix} x_1 & x_2 & \dots & x_n \end{bmatrix}^T$ and $x_i = \dfrac{det(A_i)}{det(A)}$, where $A_i$ is the matrix
    formed by replacing the $i$th column of $A$ with $\vec{b}$.

    As an example, consider the system defined by $A = \begin{bmatrix} -1 & -3 \\ 3 & 4 \end{bmatrix},
    \vec{b} = \begin{bmatrix} 11 \\ -13 \end{bmatrix}$.

    Then $A_1 = \begin{bmatrix} 11 & -3 \\ -13 & 4 \end{bmatrix}$ ($A$ but with its first column replaced by $\vec{b}$), \\
    and $A_2 = \begin{bmatrix} -1 & 11 \\ 3 & -13 \end{bmatrix}$ ($A$ but with its second column replaced by $\vec{b}$).

    Then $x_1 = \dfrac{det(A_1)}{det(A)}$ and $x_2 = \dfrac{det(A_2)}{det(A)}$. What are these determinants, and subsequently the solution to the
    system, $\vec{x}$?

    \vspace{20px}
    \begin{questions}
        \item Use Cramer's rule to solve the following system:
        $$\begin{cases}
            4x + 5y = 23 \\
            -x + 2y = 4
        \end{cases}$$
    \end{questions}

    Live demo: Proof of Cramer's Rule for $2 \times 2$ systems.

\pagebreak
\section{Closing}
\begin{questions}
    \item Based on the $2 \times 2$ proof of Cramer's rule, do you think we use it for non-invertible coefficient matrices?
    How about non-square coefficient matrices?
    \item If $det(A) = 3$ and $det(B) = \dfrac{2}{3}$, what is $det(AB^{-1})$?
    \item If $det(A) = 5$ and $det(B) = 4$ and both are $3 \times 3$ matrices, what is $det(3AB^T)$?
\end{questions}

\end{document}
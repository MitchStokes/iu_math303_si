\documentclass[11pt]{exam}
\usepackage{../commonheader}

\title{Syllabus \& Linear Systems}
\date{Week 1, Session 1}

\begin{document}
\maketitle
    
\vspace{10px}
\section{SI Logistics}
    \begin{itemize}
        \item By now, you should hopefully be in the \href{https://iu.instructure.com/courses/2278911}{Math 303 SI companion course}.
        If you're not, please let me know via email or Canvas message!
        \item Fill out the \href{https://www.when2meet.com/?25969358-aRu9i}{when2meet survey} to let me know when you're free for
        SI sessions and office hours beginning next week!
        \item Here's \href{https://asciimath.org/}{a guide to AsciiMath syntax} that you may want to bookmark. To enter "math mode"
        in a MyOpenMath window (e.g. on quizzes), just surround some math text with backticks. For example,
        \verb|`[[a,b],[c,d]]`| generates a matrix $\begin{bmatrix} a & b \\ c & d \end{bmatrix}$.
        \item To my knowledge, discussion posts use Latex instead of AsciiMath, so the above should hopefully only be relevant to quizzes/exams. 
    \end{itemize}
\pagebreak
    
\vspace{10px}
\section{Syllabus Overview}
    \vspace{10px}
    \subsection{Homework}
    \begin{itemize}
        \item For homeworks, you can retry problems twice with the answer being shown on your third attempt.
        \item If you're unable to get the problem correct even with your retries, you can select
        "Try a similar problem" to get a new version of the problem to solve. If you're able to complete this new problem,
        you will still get full credit! This means you can get full credit on every homework as long as you put in the work.
        \item Homework is due every Saturday night at 11:59 PM Eastern.
    \end{itemize}

    \vspace{10px}
    \subsection{Quizzes}
    \begin{itemize}
        \item Quizzes require submitting an answer and an explanation. 50\% credit is given for the correct answer, 50\% for the explanation.
        \item There will be a box under each quiz problem where you can type your explanation using \href{https://asciimath.org/}{AsciiMath}.
        \item For the accuracy part, you will generally have 3 attempts per question with a 30\% penalty for each incorrect attempt.
        \item Quizzes have a 3-hour time limit beginning when you start them, and are due every Monday night at 11:59 PM Eastern.
        \item \textbf{TIP:} You generally have plenty of time on quizzes. Use the extra time to check your work before submitting to make sure
        you avoid careless mistakes that cost you 30\% each! 
    \end{itemize}
    
    \vspace{10px}
    \subsection{Exams}
    \begin{itemize}
        \item There will be two exams, a midterm and a final. Both will have a 4-hour time limit and be due on Fridays at 11:59 PM Eastern.
        \item There won't be quizzes on exam weeks.
        \item The exams will be identical to quizzes in format and grading.
        \item Prior to each exam, there will be a special exam review assignment on top of the normal work for the week. This is \textit{graded} and
        generally very helpful to prepare for the exams, so make sure to do them!
    \end{itemize}
    
    \vspace{10px}
    \subsection{Discussion Boards}
    \begin{itemize}
        \item To get full credit on discussion boards, you must contribute (post or reply meaningfully) at least 5 times each week.
        \item The "Posts on Discussion Board" assignment itself is ungraded; it's just there to provide the necessary info for that week's discussion.
        \item Posts 1 and 2 will be solving your two problems from the "Posts on Discussion Board" assignment for the week. These problems will be 
        different for everyone!
        \item Post 3 will consist of finding a video related to the week's topics on YouTube and embedding it in a discussion post. Make sure no one else
        has already chosen your video (you may not get credit for this post if so)!
        \item Posts 4 and 5 should be meaningfull responses to a classmate's post. Some ideas on meaningful replies are correcting a mistake in their
        solution, offering an alternative solution method, or discussing an expansion on a topic covered by their problem.
    \end{itemize}
\pagebreak
    
\vspace{10px}
\section{hahashdsahd}

\end{document}
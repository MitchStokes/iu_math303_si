\documentclass[11pt]{exam}
\usepackage{../commonheader}
\usepackage{graphicx}

\title{Subspaces, Row/Column/Null Spaces}
\date{Week 5, Session 2}

\begin{document}
\maketitle

\section{Subspaces}
    
    \vspace{20px}
    \subsection{Subspaces}
    A collection of vectors $V$ is a  of $\mathbb{R}^n$ if whenever $a$ and $b$ are scalars and $\vec{u}$ and $\vec{v}$ are
    vectors in $V$, the linear combination $a\vec{u} + b\vec{v}$ is also in $V$.

    Another definition of a subspace $V$ is that it's a subspace of $\mathbb{R}^n$ if there exists a set of vectors that span it.
    For example, $V = span \{ \vec{u_1}, \dots, \vec{u_k} \}$.

    Additionally, if $\vec{u_1}, \dots, \vec{u_k} \in W$ where $W$ is another subspace of $\mathbb{R}^n$, then $V \subseteq W$.

    \vspace{20px}
    \subsection{Bases}
    A basis for a subspace $V$ is a linearly independent set of vectors $\{ \vec{u_1}, \dots, \vec{u_k} \}$ such that
    $V = span \{ \vec{u_1}, \dots, \vec{u_k} \}$.

    The standard basis of $\mathbb{R}^3$ is $\{ [1,0,0]^T, [0,1,0]^T, [0,0,1]^T \}$. You can create a standard basis for any $\mathbb{R}^n$
    using a similar pattern with more vectors!

    A given subspace can have many bases; a basis just needs to span the entire subspace. For example, you could span $\mathbb{R}^3$ using its standard
    basis above, but you could come up with infinitely more sets of three vectors that span the entire vector space!
    
    It turns out that bases for the same subspace must always consist of the same number of vectors though, regardless of which vectors those are.
    The number of vectors in a basis of a subspace is called its \textit{dimension}, and is denoted $dim(V)$.

    \begin{questions}
        \item Consider the following set of vectors:
        $$ V = \left\{ \begin{bmatrix} a \\ b \\ c \\ d \end{bmatrix} \in \mathbb{R}^4 : a - b = d - c \right\} $$
        Show that $V$ is a subspace, find a basis of it, and state $dim(V)$.
    \end{questions}

    A couple more theorems:
    \begin{enumerate}
        \item If we have an n-dimension set of linearly independent vectors in $\mathbb{R}^n$, then this set of vectors is a basis for $\mathbb{R}^n$.
        \item Conversely, suppose an m-dimension set of vectors spans $\mathbb{R}^n$. Then $m$ must be \textit{at least} $n$, $m \geq n$.
        \item If an n-dimension set of vectors spans $\mathbb{R}^n$, then the set of vectors is linearly independent.
        \item If $V$ is a subspace of $\mathbb{R}^n$, then its basis must have $dim(V) \leq n$.
    \end{enumerate}

\vspace{40px}
\section{Row Space, Column Space, Null Space}
    
    \vspace{20px}
    \subsection{Row Space and Column Space}
    The \textit{column space} of a matrix $A$ is the span of its columns. The \textit{row space} of $A$ is the span of its rows.
    Performing the elementary operations on a matrix doesn't change its row/column space.

    When a matrix is reduced to RREF, the nonzero rows of $A$ form a basis for its row space, $row(A)$.

    The rank of a matrix is the same as the dimension of its row space (remember, these are the number of vectors in its basis)!
    $$rank(A) = dim(row(A))$$

    TL;DR: When solving for row space, reduce the matrix to RREF, and the row space will be the span of the nonzero rows in RREF form.
    When solving for column space, reduce the matrix to RREF and determine which columns are pivots. The column space will be the span of the
    same-numbered columns in the \textit{original} matrix.

    \begin{questions}
        \item Find the rank of the following matrix and describe its column and row spaces:
        $$A = \begin{bmatrix} 1 & 2 & 1 & 3 & 2 \\ 1 & 3 & 6 & 0 & 2 \\ 3 & 7 & 8 & 6 & 6 \end{bmatrix}$$

        \item Find the rank of the following matrix and describe its column and row spaces:
        $$A = \begin{bmatrix} 1 & 2 & 1 & 3 & 2 \\ 1 & 3 & 6 & 0 & 2 \\ 1 & 2 & 1 & 3 & 2 \\ 1 & 3 & 2 & 4 & 0 \end{bmatrix}$$
    \end{questions}

    \pagebreak
    \subsection{Null Space/Kernel}
    The \textit{null space} or \textit{kernel} of a matrix is the set of vectors which carry the matrix to 0 via multiplication. In other words:
    $$null(A) = \{ \vec{x}: A \vec{x} = \vec{0} \}$$
    What does this remind us of?

    The dimension of the null space of a matrix is called the \textit{nullity}, denoted $dim(null(A))$.

    Additionally, the rank and nullity of a matrix will add up to its number of columns! This is exactly how the rank + number of basic solutions
    in a homogeneous system adds up to the number of columns.

\end{document}